\section{Die Nutzenfunktion}%
\label{sec:die_nutzenfunktion}

Eigenschaften der Konsummenge (\emph{consumption set}) $X$:
\begin{enumerate}
  \item $X \subseteq \RealNumbers^n_+$
  \item $X$ ist abgeschlossen ($\overline{X}$ ist offen)
  \item $X$ ist konvex
  \item $0 \in X$
\end{enumerate}

Es existiert die binäre Relation der Konsumentenpräferenz $\atleastasgood$ mit
$x_1 \atleastasgood x_2 \ \Leftrightarrow \ x_1$ ist mindestens genauso gut wie $x_2$.

\begin{axiom}[Vollständigkeit]
  \label{axiom:vollstaendigkeit}
  Für alle $x^1, x^2 \in X$ gilt entweder
  $x^1 \atleastasgood x^2$ oder $x^2 \atleastasgood x^1$.
\end{axiom}

\begin{axiom}[Transitivität]
  \label{axiom:transitivitaet}
  Für alle $x^1, x^2$ und $x^3 \in X$ gilt:
  wenn $x^1 \atleastasgood x^2$ und $x^2 \atleastasgood x^3$,
  dann gilt $x^1 \atleastasgood x^3$.
\end{axiom}

\begin{definition}[Präferenzrelation]
  Die binäre Relation $\atleastasgood$ auf der Konsummenge $X$ ist eine
  \emph{Präferenzrelation}, wenn es die Axiome~\ref{axiom:vollstaendigkeit}
  und~\ref{axiom:transitivitaet} erfüllt.
\end{definition}

\begin{definition}[strenge Präferenzrelation]
  Die binäre Relation $\strictlypreferredto$ auf der Konsummenge $X$ ist definiert als
  \begin{align*}
    x^1 \strictlypreferredto x^2
    \ \Leftrightarrow \
    x^1 \atleastasgood x^2 \text{ und } x^2 \notatleastasgood x^1
  \end{align*}
\end{definition}


Nutzen und Erwartungsnutzen

Siehe:
\begin{itemize}
  \item \href{https://de.wikipedia.org/wiki/Grenznutzen}{Grenznutzen}
  \item \href{https://de.wikipedia.org/wiki/Grenzrate_der_Substitution}{Grenzrate der
    Substitution}
\end{itemize}

