\section{Statische Spiele mit vollständiger Information}%
\label{sec:statische_spiele_mit_vollstandiger_information}

\begin{definition}[Rationalität]
  Rational zu handeln bedeutet, dass ein Individuum aus einer Menge von Wahlmöglichkeiten
  eine Aktion $a$ so wählt, dass $a$ mindestens genauso gut wie jede andere Aktion $b$
  ist.
  Die Güte der Aktion ist durch die Nutzenfunktion bestimmt.
\end{definition}

\subsection{Strategische Spiele}%
\label{sub:strategische_spiele}

Ein Strategienprofil ist eine Funktion, die den Spielern bestimmte Aktionen zuweist.
Bei den Spielern $\{1,2,3\}$ und den Aktionsmengen
$A_1 = \{T, B\}, A_2 = \{L, R\}, A_3 = \{I, O\}$
ist bspw. $(a_1, a_2, a_3) = (T, L, O)$ ein Strategienprofil.
Eine besondere Notation ist die Variante $(b_i, a_{-i})$, welche bedeutet, dass Spieler
$i$ die Aktion $b_i$ und \emph{nicht} $a_i$ wählt.
Wenn $b_2 = R$ ist, dann ist $(b_2, a_{-2}) = (T, R, O)$.
Alles bleibt gleich, nur Spieler $2$ ändert die Aktion.

\begin{definition}[Strategisches Spiel mit ordinaler Präferenz]
  Ein strategisches Spiel (mit ordinaler Präferenz) besteht aus:
  \begin{itemize}
    \item einer Menge von Spielern $P = \{1, \ldots, n\}$
    \item für jeden Spieler: eine Menge von Aktionen $S_i$
    \item für jeden Spieler: Präferenzen über der Menge der Aktionsprofile
      $u_i: S_1 \times \ldots \times S_n \to \RealNumbers$
  \end{itemize}
\end{definition}

\subsection{Gefangendilemma}%
\label{sub:gefangendilemma}

Zwei Verdächtige werden verhaftet und angeklagt.
Die Polizei hat nicht genügend Beweise um beide zu verurteilen, es sei denn
einer von beiden gesteht.
Die Verdächtigen werden in verschiedene Zellen gesteckt und ihnen werden die
Konsequenzen der möglichen Aktionen erklärt.
Jeder kann entweder \emph{schweigen} (kooperieren) oder \emph{gestehen}
(defektieren, gegen den anderen aussagen) und jeder Gefangene möchte möglichst
wenige Jahre im Gefängnis verbringen.
Wenn beide schweigen, dann bekommt jeder 1 Jahr wegen kleinerer Delikte.
Wenn einer schweigt und der andere gesteht, dann wird der Geständige
freigelassen und der andere bekommt die Höchststrafe von 9 Jahren Haft.
Wenn beide gestehen bekommen beide durch die Zusammenarbeit mit der Polizei (das
Geständnis) nicht die Höchststrafe, sondern jeweils nur 6 Jahre ins Gefängnis.

Die Auszahlungsmatrix (Bimatrix) sieht wie folgt aus:
\begin{center}
  \begin{tabular}{cccc}
    & & \multicolumn{2}{c}{B}\\
    & & schweigt & gesteht\\
    \cmidrule{3-4}
    \multirow{2}{*}{A}
      & schweigt & $-1, -1$ & $-9, \phantom{-}0$\\
    \cmidrule{3-4}
      & gesteht & $\phantom{-}0, -9$ & $-6, -6$\\
    \cmidrule{3-4}
  \end{tabular}
\end{center}

Die Ergebnisse haben besondere Namen:

\begin{table}[h]
  \centering
  \caption{Mögliche Erträge im Gefangenendilemma}
  \label{tab:moegliche_ertraege_im_gefangenendilemma}
  \begin{tabular}{ccr}
    \toprule
    \makecell{$T$\\ temptation\\ (Versuchung)} &
    \makecell{Verpfeifen, wenn der andere schweigt\\[1ex] (Defektion bei Kooperation)} &
    $0$\\
    \midrule
    \makecell{$R$\\ reward\\ (Belohnung)} &
    \makecell{Schweigen, wenn der andere schweigt\\[1ex] (Kooperation bei Kooperation)} &
    $-1$\\
    \midrule
    \makecell{$P$\\ punishment\\ (Bestrafung)} &
    \makecell{Verpfeifen, wenn der andere verpfeift\\[1ex] (Defektion bei Defektion)} &
    $-6$\\
    \midrule
    \makecell{$S$\\ sucker’s payoff\\ (Lohn des Gutgläubigen)} &
    \makecell{Schweigen, wenn der andere verpfeift\\[1ex] (Kooperation bei Defektion)} &
    $-9$\\
    \bottomrule
  \end{tabular}
\end{table}

Allgemein hat ein Gefangenendilemma folgende Struktur:
\begin{center}
  \begin{tabular}{cccc}
    & & \multicolumn{2}{c}{B}\\
    & & kooperiert & defektiert\\
    \cmidrule{3-4}
    \multirow{2}{*}{A}
      & kooperiert & $R, R$ & $S, T$\\
    \cmidrule{3-4}
      & defektiert & $T, S$ & $P, P$\\
    \cmidrule{3-4}
  \end{tabular}
\end{center}
Dabei müssen diese Ungleichungen gelten:
\begin{align*}
  T > R > P > S \text{ und } 2R > T + S
\end{align*}
Die erste geht aus den einzelnen Präferenzen hervor und die zweite sorgt dafür, dass die
Verdächtigen sich bei Wiederholung nicht gegenseitig ausbeuten können.
Bei Wiederholung könnten sie abwechselnd kooperieren und defektieren und würden auf Dauer
mehr Nutzen haben.
Diese Möglichkeit muss ausgeschlossen werden.

\subsection{Bach oder Strawinski}%
\label{sub:bach_oder_strawinski}

Beim Gefangenendilemma ist das Hauptproblem, ob beide Spieler kooperieren.
Im folgenden Spiel sind sich beide einig, dass es besser ist zu kooperieren, aber sie sind
sich nicht über die Aktion einig.

Zwei Menschen wollen zusammen den Abend verbringen.
Zwei Konzerte stehen zur Verfügung: Bach oder Strawinski.
Eine Person präferiert Bach, die andere Strawinski.
Wenn sie auf verschiedene Konzerte gehen, sind sie beide gleich unglücklich, weil sie den
Abend nicht mit der anderen Person verbringen.
\begin{center}
  \begin{tabular}{cccc}
    & & \multicolumn{2}{c}{B}\\
    & & Bach & Strawinski\\
    \cmidrule{3-4}
    \multirow{2}{*}{A}
      & Bach & $2, 1$ & $0, 0$\\
    \cmidrule{3-4}
      & Strawinski & $0, 0$ & $1, 2$\\
    \cmidrule{3-4}
  \end{tabular}
\end{center}

\subsection{Matching Pennies}%
\label{sub:matching_pennies}

Zwei Spieler wählen gleichzeitig ob sie Kopf oder Zahl einer Münze zeigen.
Wenn die Münzen die gleiche Seite zeigen, dann zahlt Spieler 2 einen Euro an Spieler 1.
Wenn sie verschiedene Seiten zeigen, dann zahlt Spieler 1 einen Euro an Spieler 2.

\begin{center}
  \begin{tabular}{cccc}
    & & \multicolumn{2}{c}{Spieler 2}\\
    & & Kopf & Zahl\\
    \cmidrule{3-4}
    \multirow{2}{*}{Spieler 1}
      & Kopf & $\phantom{-}1, -1$ & $-1, \phantom{-}1$\\
    \cmidrule{3-4}
      & Zahl & $-1, \phantom{-}1$ & $\phantom{-}1, -1$\\
    \cmidrule{3-4}
  \end{tabular}
\end{center}

\subsection{Hirschjagd (Stag hunt)}%
\label{sub:hirschjagd_stag_hunt}

Eine Jägerin einer Gruppe hat zwei Optionen: entweder konzentriert sie sich auf die
Verfolgung eines Hirsches oder sie schießt vielleicht einen Hasen.
Wenn alle Jäger den Hirsch verfolgen, dann fangen sie ihn und teilen ihn gleich auf.
Wenn ein Jäger sich auf den Hasen konzentriert, dann entwischt der Hirsch und der Hase
gehört nur dem abtrünnigen Jäger.
Jeder Jäger präferiert einen Teil des Hirsches gegenüber einem Hasen.

Das Spiel besteht demnach aus den folgenden Elementen:
\begin{description}
  \item[Spieler] Die Jäger.
  \item[Aktionen] jeder Jäger hat die Aktionen $\{\emph{Hirsch}, \emph{Hase}\}$.
  \item[Präferenzen] Für jeden Jäger ist das Aktionsprofil $(\emph{Hirsch}, \ldots,
    \emph{Hirsch})$, also alle wählen Hirsch, am wertvollsten.
    Danach kommt für jeden Jäger das Profil, in dem er \emph{Hase} wählt, gefolgt von
    allen Profilen, in denen er \emph{Hirsch} wählt und irgendein anderer \emph{Hase}
    wählt (sodass er leer ausgeht).
\end{description}
Für zwei Spieler sieht das Spiel wie folgt aus:
\begin{center}
  \begin{tabular}{cccc}
    & & \multicolumn{2}{c}{Spieler 2}\\
    & & Hirsch & Hase\\
    \cmidrule{3-4}
    \multirow{2}{*}{Spieler 1}
      & Hirsch & $2, 2$ & $0,1$\\
    \cmidrule{3-4}
      & Hase   & $1, 0$ & $1,1$\\
    \cmidrule{3-4}
  \end{tabular}
\end{center}

\subsection{Nash-Gleichgewicht}%
\label{sub:nash_gleichgewicht}

Ein Gleichgewicht ist eine stabile Situation, in der kein Spieler einen Anreiz hat, sein
Verhalten zu verändern.

Ein \emph{Nash-Gleichgewicht} ist ein Aktionsprofil $a^* = (a^*_1, \ldots, a^*_n)$,
bei dem kein Spieler $i$ eine bessere Aktion als $a^*_i$ wählen kann,
sofern die anderen Spieler $j$ an $a^*_j$ festhalten.

\begin{definition}[Nash-Gleichgewicht eines strategischen Spiels mit ordinaler Präferenz]
  Das Aktionsprofil $a^*$ ist genau dann ein Nash-Gleichgewicht,
  wenn es für jeden Spieler $i$ und für jede Aktion $a_i$ von Spieler $i$
  bezüglich der Präferenzen von $i$
  mindestens genauso gut wie das Aktionsprofil $(a_i, a^*_{-i})$ ist,
  bei dem Spieler $i$ die Aktion $a_i$ statt $a^*_{-i}$ wählt,
  während alle anderen Spieler $j$ jeweils $a^*_j$ wählen.

  Äquivalent gilt für jeden Spieler $i$
  \begin{align*}
    u_i(a^*) \geq u_i(a_i, a^*_{-i}) \quad \text{für jede Aktion $a_i$ von Spieler $i$.}
  \end{align*}
\end{definition}

\subsection{Beste-Antwort-Funktionen}%
\label{sub:beste_antwort_funktionen}

\begin{definition}[Beste-Antwort-Funktion]
  Die Menge der besten Aktionen von Spieler $i$ gegenüber den Aktionen $a_{-i}$ der
  anderen Spieler wird mit $B_i(a_{-i})$ notiert.
  Genau definiert wird sie als:
  \begin{align*}
    B_i(a_{-i}) & = \left\{
      a_i \in A_i :
      u_i(a_i, a_{-i}) \geq u_i(a'_i, a_{-i})
      \text{ für alle $a'_i \in A_i$}
    \right\}
  \end{align*}
  Jede Aktion in $B_i(a_{-i})$ ist mindestens so gut für Spieler $i$ wie jede andere
  Aktion von Spieler $i$, wenn die Aktionen der anderen Spieler $a_{-i}$ sind.

  Wenn jeder andere Spieler an der Aktion $a_{-i}$ festhält, dann kann Spieler $i$ nicht
  mehr erreichen, als durch eine Aktion aus $B_i(a_{-i})$.
\end{definition}

Im Spiel \emph{Bach oder Strawinski} ist
$B_1(\emph{Bach}) = \{ \emph{Bach} \}$ und
$B_1(\emph{Strawinski}) = \{ \emph{Strawinski} \}$.

Mit der Besten-Antwort-Funktion (auch Reaktionsfunktion genannt) lässt sich auch das
Nash-Gleichgewicht definieren:
das Aktionsprofil $a^*$ ist genau dann ein Nash-Gleichgewicht eines strategischen Spiels
mit ordinaler Präferenz, wenn jede Aktion der Spieler eine beste Antwort auf die Aktionen
der anderen Spieler ist:
\begin{align}
  \label{eq:beste_antwort_nash_gg}
  a^*_i \in B_i(a^*_{-i}) \text{ für alle Spieler $i$}
\end{align}
Wenn die Menge der besten Antworten aus genau einem Element besteht
($B_i(a_{-i}) = \{ b_i(a_{-i}) \}$),
dann wird dieses Element als $b_i(a_{-i})$ notiert.
Dann gilt:
\begin{align*}
  a^*_i = b_i(a^*_{-i}) \text{ für alle Spieler $i$}
\end{align*}

Mit der Beste-Antwort-Funktion lässt sich das Nash-Gleichgewicht finden.
Finde dazu die besten Antworten aller Spieler und finde dann die Aktionsprofile,
die~\ref{eq:beste_antwort_nash_gg} erfüllen.

\subsection{Dominanz}%
\label{sub:dominanz}

\paragraph{Dominierte Aktionen}%
\label{par:dominierte_aktionen}

\begin{definition}[Strikte Domination]
  \label{def:strikte_domination}
  Eine Aktion $\hat{a}_i$ von Spieler $i$ wird \emph{strikt dominiert},
  wenn es eine andere Aktion $\tilde{a}_i \in A_i$ gibt,
  so dass $\tilde{a}_i$ zu einer strikt größeren Auszahlung führt als $\hat{a}_i$
  — unabhängig davon, welche Aktion von den Gegenspielern gewählt werden:
  \begin{align*}
    u_i(\hat{a}_i, a_{-i}) < u_i(\tilde{a}_i, a_{-i})
    \quad
    \forall a_{-i} \text{ der anderen Spieler}
  \end{align*}
\end{definition}

Im Gefangenendilemma wird \emph{Kooperieren} strikt von \emph{Defektieren} dominiert, weil
$T > R$ und $P > S$ gilt.

Es gilt Rationalität bei strikter Dominanz: ein rationaler Spieler wird niemals eine
strikt dominierte Strategie spielen.

\begin{definition}[Schwache Dominanz]
  \label{def:schwache_dominanz}
  Eine Aktion $\hat{a}_i$ von Spieler $i$ wird \emph{schwach dominiert}, wenn es eine
  andere Aktion $\tilde{a}_i \in A_i$ gibt, so dass gilt:
  \begin{align*}
    u_i(\hat{a}_i, a_{-i}) \leq u_i(\tilde{a}_i, a_{-i})
    \quad \forall a_{-i} \text{ der anderen Spieler}
  \end{align*}
  und
  \begin{align*}
    u_i(\hat{a}_i, a_{-i}) < u_i(\tilde{a}_i, a_{-i})
    \quad \text{für mindestens ein $a_{-i}$ der anderen Spieler}
  \end{align*}
\end{definition}

Ein rationaler Spieler kann durchaus eine schwach dominierte Strategie spielen.

\begin{definition}[dominante Strategie]
  \label{def:strikt_dominante_strategie}
  Eine Strategie $s^*_i$ von Spieler $i$ ist eine \emph{strikt dominante} Strategie, falls
  sie alle anderen Strategien von Spieler $i$ strikt dominiert:
  \begin{align*}
    u_i(s^*_i, s) > u_i(s_i, s) \qquad
    \forall s_i \in S_i \setminus \{s^*_i\},
    \forall s \in S_{j \neq i}
  \end{align*}
\end{definition}

Wenn eine strikt dominante Strategie $s^*_i$ existiert, wird sie ein rationaler Spieler
immer wählen.

\begin{definition}[Iterative Eliminierung strikt dominierter Strategien]
  Wiederhole solange, bis sich nichts mehr ändert:
  finde eine dominierte Strategie und eliminiere sie.
\end{definition}

Beispiel: 2 Spieler, 4 Strategien

\begin{center}
  \begin{tabular}{cccccc}
    & & \multicolumn{4}{c}{B}\\
    & & $b_1$ & $b_2$ & $b_3$ & $b_4$\\
    \cmidrule{3-6}
    \multirow{4}{*}{A} & $a_1$ & $0,7$ & $2,5$ & $7,0$ & $0,1$\\
    \cmidrule{3-6}
    & $a_2$ & $5,2$ & $3,3$ & $5,2$ & $0,1$\\
    \cmidrule{3-6}
    & $a_3$ & $7,0$ & $2,5$ & $0,7$ & $0,1$\\
    \cmidrule{3-6}
    & $a_4$ & $0,0$ & $0,-2$ & $0,0$ & $10,-3$\\
    \cmidrule{3-6}
  \end{tabular}
\end{center}

Egal welche Strategie Spieler A wählt, für B ist $b_2$ immer besser als $b_4$, also wird
$b_4$ eliminiert:

\begin{center}
  \begin{tabular}{ccccc}
    & & \multicolumn{3}{c}{B}\\
    & & $b_1$ & $b_2$ & $b_3$\\
    \cmidrule{3-5}
    \multirow{4}{*}{A} & $a_1$ & $0,7$ & $2,5$ & $7,0$\\
    \cmidrule{3-5}
    & $a_2$ & $5,2$ & $3,3$ & $5,2$\\
    \cmidrule{3-5}
    & $a_3$ & $7,0$ & $2,5$ & $0,7$\\
    \cmidrule{3-5}
    & $a_4$ & $0,0$ & $0,-2$ & $0,0$\\
    \cmidrule{3-5}
  \end{tabular}
\end{center}

Jetzt ist es egal, welche Strategie Spieler B wählt, für A ist $a_2$ immer besser als
$a_4$ und damit wird $a_4$ eliminiert:

\begin{center}
  \begin{tabular}{ccccc}
    & & \multicolumn{3}{c}{B}\\
    & & $b_1$ & $b_2$ & $b_3$\\
    \cmidrule{3-5}
    \multirow{4}{*}{A} & $a_1$ & $0,7$ & $2,5$ & $7,0$\\
    \cmidrule{3-5}
    & $a_2$ & $5,2$ & $3,3$ & $5,2$\\
    \cmidrule{3-5}
    & $a_3$ & $7,0$ & $2,5$ & $0,7$\\
    \cmidrule{3-5}
  \end{tabular}
\end{center}

\subsection{Gemischte Strategien}%
\label{sub:gemischte_strategien}

Wir betrachten das Spiel \emph{Matching Pennies} aus \ref{sub:matching_pennies} an.
Angenommen Spieler 2 wählt jede der möglichen Aktionen mit der Wahrscheinlichkeit
$\frac{1}{2}$.
Wenn Spieler 1 mit der Wahrscheinlichkeit $p$ \emph{Kopf} wählt und \emph{Zahl} mit der
Wahrscheinlichkeit $1-p$, dann treten die Ereignisse
(\emph{Kopf}, \emph{Kopf}) und (\emph{Kopf}, \emph{Zahl})
mit der Wahrscheinlichkeit $\frac{1}{2}p$ auf und die Ereignisse
(\emph{Zahl}, \emph{Kopf}) und (\emph{Zahl}, \emph{Zahl})
mit der Wahrscheinlichkeit $\frac{1}{2}(1-p)$.
Damit ist die Wahrscheinlichkeit für die Ereignisse
(\emph{Kopf}, \emph{Kopf}) und (\emph{Zahl}, \emph{Zahl}),
bei denen Spieler 1 das Geld gewinnt,
$\frac{1}{2}p + \frac{1}{2}(1-p) = \frac{1}{2}$.
Die Wahrscheinlichkeit für die anderen Ereignisse, bei denen sie Geld verliert, beträgt
ebenfalls $\frac{1}{2}$.
Wichtig ist hierbei, dass diese Wahrscheinlichkeiten nicht mehr von $p$ abhängen, obwohl
Spieler 1 das eigene Verhalten von $p$ abhängig macht.
Damit ist jeder Wert von $p$ optimal.
Spieler 1 kann es daher nicht besser treffen, als \emph{Kopf} und \emph{Zahl} jeweils mit
der Wahrscheinlichkeit $\frac{1}{2}$ zu wählen.

\begin{definition}[Gemischte Strategie]
  Eine \emph{gemischte Strategie} eines Spielers in einem strategischem Spiel ist eine
  Wahrscheinlichkeitsverteilung über den Aktionen des Spielers.

  Notation: $α_i(a_i)$ ist die Wahrscheinlichkeit von Aktion $a_i$ von Spieler $i$.
  Verkürzte Notation: $\left( \frac{1}{3}, \frac{2}{3} \right)$ für die
  Wahrscheinlichkeiten der Aktionen anhand der Reihenfolge der Tabelle des Spiels in
  Normalform.
\end{definition}

Wenn eine Aktion die Wahrscheinlichkeit $1$ bekommt, dann ist diese gemischte Strategie
eine \emph{reine Strategie}.

\begin{definition}[Nash-Gleichgewicht einer gemischten Strategie eines strategischen
  Spiels mit vNM-Präferenzen]
  Das gemischte Strategieprofil $α^*$ in einem strategischen Spiel mit vNM-Präferenzen ist
  ein \emph{Nash-Gleichgewicht (einer gemischten Strategie)},
  wenn für jeden Spieler $i$ und jede gemischte Strategie $α_i$ von Spieler $i$
  der Erwartungsnutzen von Spieler $i$ von $α^*$ mindestens so hoch ist
  wie der Erwartungsnutzen von Spieler $i$ von $(α_i, α^*_{-i})$
  bzgl. der Nutzenfunktion dessen Erwartungswert die Präferenzen von Lotterien
  von Spieler $i$ repräsentiert.
  Äquivalent gilt, dass für jeden Spieler $i$
  \begin{align*}
    U_i(α^*) \geq U_i(α_i, α^*_{-1}) \text{ für jede gemischte Strategie $α_i$ von Spieler
    $i$,}
  \end{align*}
  gilt, wobei $U_i(α)$ der Erwartungsnutzen von Spieler $i$ zu dem gemischten
  Strategieprofil $α$ ist.
\end{definition}

Der allgemeine Fall in einem Spiel für zwei Spieler mit den Aktionen $T$ und $B$ für
Spieler 1 und $L$ und $R$ für Spieler 2, wobei $p = α_1(T)$ und $q = α_2(L) = q$ ist:
\begin{center}
  \begin{tabular}{cccc}
    & & \multicolumn{2}{c}{2}\\
    & & $L$ & $R$\\
    \cmidrule{2-3}
    \multirow{2}{*}{1}
      & $T$ & $pq$ & $p(1-q)$\\
    \cmidrule{2-3}
      & $B$ & $(1-p)q$ & $(1-p)(1-q)$\\
    \cmidrule{2-3}
  \end{tabular}
\end{center}

Der Erwartungsnutzen für Spieler 1 ist:
\begin{align*}
  & pq \cdot u_1(T,L)
  + p(1-q) \cdot u_1(T,R)
  + (1-p)q \cdot u_1(B,L)
  + (1-p)(1-q) \cdot u_1(B,R)\\
  & =
    p     \left[ q \cdot u_1(T,L) + (1-q) \cdot u_1(T,R) \right]
  + (1-p) \left[ q \cdot u_1(B,L) + (1-q) \cdot u_1(B,R) \right]
\end{align*}

Der erste Term in den eckigen Klammern ist der Erwartungsnutzen von Spieler 1,
wenn sie die reine Strategie \emph{spiele $T$ mit Wahrscheinlichkeit $1$}
und Spieler die gemischte Strategie $α_2$ wählt;
der zweite Term in den eckigen Klammern ist der Erwartungsnutzen von Spieler 1,
wenn sie die reine Strategie \emph{spiele $B$ mit Wahrscheinlichkeit $1$}
und Spieler 2 die gemischte Strategie $α_2$ wählt.
Seien $E_1(T, α_2)$ und $E_1(B, α_2)$ diese beiden Erwartungsnutzen.
Dann ist der Erwartungsnutzen von Spieler 1 zum gemischten Strategiepaar $(α_1, α_2)$
\begin{align*}
  p E_1(T, α_2) + (1-p) E_1(B, α_2).
\end{align*}
Das heißt, dass der Erwartungsnutzen von Spieler 1 zum gemischten Strategiepaar
$(α_1, α_2)$ das gewichtete Mittel der Erwartungsnutzen von $T$ und $B$ ist,
wenn Spieler 2 die gemischte Strategie $α_2$ spielt,
wobei die Gewichte die Wahrscheinlichkeiten von $T$ und $B$ aus $α_1$ sind.

Genauer heißt das, dass der Erwartungsnutzen von Spieler 1 zur gemischten Strategie von
Spieler 2 eine \emph{lineare} Funktion in $p$ ist.
Für den Fall $E_1(T, α_2) > E_1(B, α_2)$ sieht das wie folgt aus:
\begin{center}
  \begin{tikzpicture}
    \draw (0,0) -- (8,0);
    \draw (0,0) -- (0,5);
    \draw (8,0) -- (8,5);

    \draw (0,0) [below] node {$0$};
    \draw (8,0) [below] node {$1$};

    \draw (0,1) -- (8,4);
    \draw (0,1) node [left] {$E_1(B, α_2)$};
    \draw (8,4) node [right] {$E_1(T, α_2)$};

    \draw (3,0) [below] node {$p$};
    \draw (0,2.125) [left] node {$p E_1(T, α_2) + (1-p) E_1(B, α_2)$};
    \draw [dashed] (3,0) -- (3, 2.125);
    \draw [dashed] (0, 2.125) -- (3, 2.125);

    \draw (0,4) [left] node {Erwartungsnutzen von Spieler 1};
  \end{tikzpicture}
\end{center}
