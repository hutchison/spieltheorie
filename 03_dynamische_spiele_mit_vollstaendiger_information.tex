\section{Dynamische Spiele mit vollständiger Information}%
\label{sec:dynamische_spiele_mit_vollstandiger_information}

\begin{definition}[Extensives Spiel mit perfekter Information]
  Ein extensives Spiel mit perfekter Information besteht aus
  \begin{itemize}
    \item einer Menge von \emph{Spielern}
    \item einer Menge von Folgen (\emph{terminalen Geschichten}) mit der Eigenschaft, dass
      keine Folge eine echte Teilgeschichte einer anderen Folge ist
    \item einer Funktion (der \emph{Spielerfunktion}), die einem Spieler jede echte
      Teilgeschichte irgendeiner terminalen Geschichte zuweist
    \item für jeden Spieler: Präferenzen über der Menge der terminalen Geschichten
  \end{itemize}
\end{definition}

\begin{definition}[Nash-Gleichgewicht eines extensiven Spiels mit perfekter Information]
  Das Strategieprofil $s^*$ in einem extensiven Spiel mit perfekter Information ist genau
  dann ein Nash-Gleichgewicht, wenn für jeden Spieler $i$ und jede Strategie $r_i$ von
  Spieler $i$ die von $s^*$ erzeugte terminale Geschichte $O(s^*)$ (abhängig von den
  Präferenzen von Spieler $i$) mindestens genauso gut ist, wie die von dem Strategieprofil
  $(r_i, s^*_{-i})$ erzeugte terminale Geschichte $O(r_i, s^*_{-i})$, in der $r_i$ von
  Spieler $i$ gewählt wurde, während jeder andere Spieler $j$ das Strategieprofil $s^*_j$
  gewählt hat.

  Äquivalent gilt für jeden Spieler $i$:
  \begin{align*}
    u_i(O(s^*)) \geq u_i(O(r_i, s^*_{-i})) \quad
      \text{für jede Strategie $r_i$ von Spieler $i$}
  \end{align*}
  wobei $u_i$ die Nutzenfunktion ist, die die Präferenzen von Spieler $i$ repräsentiert
  und $O$ die Auszahlungsfunktion des Spiels ist.
\end{definition}
