\subsection{Serie 2}%
\label{sub:serie_2}

\paragraph{Aufgabe 1}%
\label{par:serie_2_aufgabe_1}

Paula und Anton planen gemeinsam eine WG in München zu gründen.
Im Zuge der Planungen weist Paula darauf hin, dass sowohl sie als auch Anton ja vermutlich
keine Lust hätten, regelmäßig die Küche sauber zu machen, obwohl sie beide gern eine
saubere Küche hätten.
Anton bejaht die Präferenzen, glaubt aber, dass sich das schon regeln werde.
Was glauben Sie?

\begin{enumerate}
  \item Stellen Sie die Interaktion \textit{Küche putzen} als Normalformspiel dar und
    beschreiben Sie die Annahmen, die Sie dabei treffen. (Vernachlässigen Sie bitte für
    den Moment, dass sich das Problem jede Woche wiederholt.)

  \item Bestimmen Sie alle dominanten Strategien.

  \item Bestimmen Sie alle rationalisierbaren Strategien.

  \item Bestimmen Sie alle Nash-Gleichgewichte (i.e., alle Strategien, die wechselseitig
    beste Antworten darstellen) in reinen Strategien.
\end{enumerate}

\subparagraph{Lösung}%

\begin{enumerate}
  \item Es gibt zwei Spieler: \textit{Anton} und \textit{Paula}.
    Jeder Spieler hat zwei Aktionen: \textit{putzen $(p)$}
    und \textit{nicht putzen $(\neg p)$}.
    Beide Spieler finden eine saubere Küche immer besser als eine schmutzige, d.\,h. es
    ist immer besser, wenn irgendjemand putzt, als wenn niemand putzt.
    Weiterhin finden beide Spieler \textit{nicht putzen} besser als \textit{putzen}.
    Für die möglichen Strategienprofile
    $(p,p), (p, \neg p), (\neg p, p), (\neg p, \neg p)$
    ergeben sich für Anton folgende Bedingungen bzgl. der Auszahlungen:
    \begin{align*}
      u_A(p,p) & > u_A(\neg p, \neg p)\\
      u_A(p,\neg p) & > u_A(\neg p, \neg p)\\
      u_A(\neg p, p) & > u_A(\neg p, \neg p)
        & \text{(Eine saubere Küche ist besser als eine schmutzige.)}\\
      u_A(\neg p, p) & > u_A(p, p)\\
      u_A(\neg p, p) & > u_A(p, \neg p)
        & \text{(Eine saubere Küche ohne putzen zu müssen ist besser als alles andere.)}\\
      u_A(p, p) & > u_A(p, \neg p)
        & \text{(Gemeinsam putzen ist besser als ausgenutzt zu werden.)}
    \end{align*}
    Daraus ergibt sich folgende Rangordnung der Auszahlungen:
    \begin{align*}
      u_A(\neg p, p) > u_A(p, p) > u_A(p, \neg p) > u_A(\neg p, \neg p)
      \intertext{bzw. in Zahlen:}
      3 > 2 > 1 > 0
    \end{align*}
    Die Auszahlungen für Paula sind symmetrisch:
    \begin{align*}
      u_P(p, \neg p) > u_P(p, p) > u_P(\neg p, p) > u_P(\neg p, \neg p)
    \end{align*}
    Damit ergibt sich folgende Auszahlungsbimatrix:
    \begin{center}
      \begin{tabular}{cccc}
        & & \multicolumn{2}{c}{Paula}\\
        & & $p$ & $\neg p$\\
        \cmidrule{3-4}
        \multirow{2}{*}{Anton}
        & $p$       & $2,2$ & $1,3$ \\
        \cmidrule{3-4}
        & $\neg p$  & $3,1$ & $0,0$\\
        \cmidrule{3-4}
      \end{tabular}
    \end{center}

  \item Es existieren keine dominanten Strategien.
    Für beide Spieler ist $p$ nicht immer besser als $\neg p$
    und andersherum ist $\neg p$ nicht immer besser als $p$.

  \item Betrachten wir Antons Sicht: wenn Paula $\neg p$ spielt,
    dann ist seine beste Antwort darauf $p$ zu spielen.
    Wenn Paula $p$ spielt, dann ist seine beste Antwort darauf $\neg p$ zu spielen.
    Aus Paulas Sicht ist die Argumentation gleich.
    Damit sind $(p, \neg p)$ und $(\neg p, p)$ rationalisierbar.

  \item $\text{NGG} = \{(p, \neg p), (\neg p, p)\}$
\end{enumerate}

\paragraph{Aufgabe 2}%
\label{par:serie_2_aufgabe_2}

Wenn Anton sich morgens in die Uni aufmacht, kann er sowohl mit der Tram fahren als auch
zu Fuß gehen.
Anton fährt am liebsten Tram, wenn er einen Sitzplatz bekommt, geht aber lieber zu Fuß,
als in der vollen Tram zu stehen.
Dasselbe gilt für Paula und Julia, die nebenan wohnt.
Wie es der Zufall will scheinen aber jeden Morgen nur 2 Plätze in der Tram frei zu sein,
wenn die 3 an der Haltestelle ankommen.

\begin{enumerate}
  \item Beschreiben Sie die Situation \textit{Weg in die Uni} an einem bestimmten Morgen
    für Anton, Paula und Julia als Normalformspiel.

  \item Bestimmen Sie alle Nash-Gleichgewichte in reinen Strategien.
\end{enumerate}

\subparagraph{Lösung}%

\begin{enumerate}
  \item Es gibt 3 Spieler: \textit{Anton}, \textit{Paula} und \textit{Julia}.
    Jeder Spieler hat 2 mögliche Aktionen: \textit{Tram fahren $(T)$} und \textit{zu Fuß
    gehen $(F)$}.
    Die einzelnen gewählten Aktionen führen zu drei verschiedenen Zuständen für die
    einzelnen Spieler:
    \begin{itemize}
      \item \textit{Gehen}, wenn der Spieler $F$ spielt
      \item \textit{Stehen}, wenn alle Spieler $T$ spielen und die anderen Spieler die
        Plätze besetzen
      \item \textit{Sitzen}, wenn der Spieler $T$ spielt und mindestens ein anderer
        Spieler $F$ spielt
    \end{itemize}

    Es gilt: \textit{Sitzen} ist besser als \textit{Gehen} ist besser als \textit{Stehen},
    sodass für alle Spieler $i$ gilt:
    \begin{align*}
      u_i(\textit{Sitzen}) & > u_i(\textit{Gehen}) > u_i(\textit{Stehen})
    \end{align*}
    Daher seien folgende Auszahlungen für die Spieler definiert:
    \begin{align*}
      u_i(\textit{Sitzen}) & = 3\\
      u_i(\textit{Gehen})  & = 2\\
      u_i(\textit{Stehen}) & = 1
    \end{align*}
    Wenn alle drei Spieler $T$ spielen und nur zwei Plätze zur Verfügung stehen, dann
    müssen nicht alle Spieler stehen, sondern zwei können sitzen und nur einer muss
    stehen.
    Für diesen Fall ergibt sich die Auszahlung aus dem Erwartungsnutzen:
    \begin{align*}
      EU_i\left[(T,T,T)\right] &
        = \frac{2}{3} u_i(\textit{Sitzen}) + \frac{1}{3} u_i(\textit{Stehen})
        = \frac{7}{3}
    \end{align*}
    Die Auszahlungen müsste man eigentlich als $2 \times 2 \times 2$ Würfel darstellen.
    Der Übersichtlichkeit halber soll sie scheibenweise aufgeschrieben werden:

    \begin{center}
      \begin{tabular}{cccc}
        Julia spielt $T$ & & \multicolumn{2}{c}{Paula}\\
        & & $T$ & $F$\\
        \cmidrule{3-4}
        \multirow{2}{*}{Anton}
        & $T$
        & $\underline{\frac{7}{3}},\underline{\frac{7}{3}},\underline{\frac{7}{3}}$
        & $\underline{3},2,\underline{3}$\\
        \cmidrule{3-4}
        & $F$ & $2,\underline{3},\underline{3}$ & $2,2,\underline{3}$\\
        \cmidrule{3-4}
      \end{tabular}
      \qquad
      \begin{tabular}{cccc}
        Julia spielt $F$ & & \multicolumn{2}{c}{Paula}\\
        & & $T$ & $F$\\
        \cmidrule{3-4}
        \multirow{2}{*}{Anton}
        & $T$
        & $\underline{3},\underline{3},2$
        & $\underline{3},2,2$\\
        \cmidrule{3-4}
        & $F$ & $2,\underline{3},2$ & $2,2,2$\\
        \cmidrule{3-4}
      \end{tabular}
    \end{center}

  \item $\text{NGG} = \{(T,T,T)\}$
\end{enumerate}

\paragraph{Aufgabe 3}%
\label{par:serie_2_aufgabe_3}

Betrachten Sie folgendes Zwei-Personen-Spiel in strategischer Form.
Spieler 1 wählt die Zeile (Top, Middle, oder Bottom) und Spieler 2 wählt gleichzeitig die
Spalte (Left, Centre, oder Right).
Jede Zelle in der Spielmatrix spezifiziert die Auszahlungen der betreffenden
Strategiekombinationen, d.\,h. ein Ergebnis des Spieles.

\begin{center}
  \begin{tabular}{ccccc}
    & & \multicolumn{3}{c}{Spieler 2}\\
    & & Left & Centre & Right\\
    \cmidrule{3-5}
    \multirow{3}{*}{Spieler 1}
    & Top & $7,2$ & $6,5$ & $3,3$\\
    \cmidrule{3-5}
    & Middle & $3,5$ & $3,3$ & $5,6$\\
    \cmidrule{3-5}
    & Bottom & $5,6$ & $2,6$ & $4,7$\\
    \cmidrule{3-5}
  \end{tabular}
\end{center}

\begin{enumerate}
  \item Was bedeutet der Begriff \textit{dominierte Strategie} im Kontext dieses Spiels?
    Gibt es dominierte Strategien in diesem Spiel?

  \item Was bedeutet das Konzept \textit{Nash-Gleichgewicht (NGG)}?
    Was sind die reinen NGGe dieses Spieles?
\end{enumerate}

\subparagraph{Lösung}%

\begin{enumerate}
  \item Für Spieler 2 sind die Auszahlungen von Left immer kleiner als die von Right, egal
    was Spieler 1 spielt.
    Daher ist Left eine von Right dominierte Strategie.
    Sie ist sogar strikt dominiert, weil die Auszahlungen immer echt kleiner als die von
    Right sind.

  \item Ein Nash-Gleichgewicht ist eine Kombination von Strategien, wobei jeder Spieler
    genau eine Strategie wählt, von der aus für keinen Spieler sinnvoll ist, von der
    gewählten Strategie als einziger abzuweichen.

    Mittels \textit{iterativer Elimination strikt dominierter Strategien} und der
    Bestimmung der jeweils besten Antworten lassen sich die Nash-Gleichgewichte finden.

    Da Left eine strikt dominierte Strategie ist, wird Spieler 2 sie niemals spielen,
    sodass folgendes Spiel übrig bleibt:

    \begin{center}
      \begin{tabular}{cccc}
        & & \multicolumn{2}{c}{Spieler 2}\\
        & & Centre & Right\\
        \cmidrule{3-4}
        \multirow{3}{*}{Spieler 1}
        & Top & $6,5$ & $3,3$\\
        \cmidrule{3-4}
        & Middle & $3,3$ & $5,6$\\
        \cmidrule{3-4}
        & Bottom & $2,6$ & $4,7$\\
        \cmidrule{3-4}
      \end{tabular}
    \end{center}

    In diesem reduzierten Spiel wird Bottom strikt von Middle dominiert, sodass Spieler 1
    niemals Bottom spielen wird. Es bleibt:

    \begin{center}
      \begin{tabular}{cccc}
        & & \multicolumn{2}{c}{Spieler 2}\\
        & & Centre & Right\\
        \cmidrule{3-4}
        \multirow{3}{*}{Spieler 1}
        & Top & $6,5$ & $3,3$\\
        \cmidrule{3-4}
        & Middle & $3,3$ & $5,6$\\
        \cmidrule{3-4}
      \end{tabular}
    \end{center}

    Für die Nash-Gleichgewichte werden die jeweils besten Antworten bestimmt:
    \begin{align*}
      BA_1(\text{Centre}) & = \text{Top}\\
      BA_1(\text{Right}) & = \text{Middle}\\
      BA_2(\text{Top}) & = \text{Centre}\\
      BA_2(\text{Middle}) & = \text{Right}
    \end{align*}
    Damit ist $\text{NGG} = \{(\text{Top, Centre}), (\text{Middle, Right})\}$
\end{enumerate}

\paragraph{Aufgabe 4}%
\label{par:serie_2_aufgabe_4}

Zwei Spieler haben, wie im Gefangenendilemma, die beiden möglichen Aktionen $C$ und $D$.
Die folgende Matrix beschreibt die \textit{Geldbeträge}, welche die Spieler bei der Wahl
des entsprechenden Aktionenprofils erhalten.

\begin{center}
  \begin{tabular}{ccc}
    & $C$ & $D$\\
    \cmidrule{2-3}
    $C$ & $2,2$ & $0,3$\\
    \cmidrule{2-3}
    $D$ & $3,0$ & $1,1$\\
    \cmidrule{2-3}
  \end{tabular}
\end{center}

Die Spieler sind jedoch nicht (unbedingt) egoistisch; ihre Präferenzen hängen auch von dem
Geldbetrag ab, den der Mitspieler erhält.
So sind die Präferenzen von Spieler $i$ durch eine Nutzenfunktion der Form
\begin{align*}
  m_i + \alpha m_j, \quad \alpha \geq 0
\end{align*}
beschrieben, wobei $m_i$ der Geldbetrag ist, den Spieler $i$ erhält, und $m_j$ der
Geldbetrag ist, den der Mitspieler $j \neq i$ erhält.
Für welche Werte von $\alpha$ hat das resultierende Spiel die gleiche Auszahlungsstruktur
wie das Gefangenendilemma?
Bestimmen Sie die Nash-Gleichgewichte des Spieles für die Werte von $\alpha$, für die
dieses nicht der Fall ist.

\subparagraph{Lösung}%
Siehe~\ref{sub:gefangendilemma}.
Ohne die besondere Nutzenfunktion haben die Erträge folgende Werte:
\begin{align*}
  T = 3 \quad R = 2 \quad P = 1 \quad S = 0
\end{align*}
Damit das Spiel ein Gefangenendilemma wird, muss $T > R > P > S$ gelten.
Durch die Nutzenfunktion ergibt sich folgende Bimatrix der Auszahlungen:
\begin{center}
  \begin{tabular}{ccc}
    & $C$ & $D$\\
    \cmidrule{2-3}
    $C$ & $2+2\alpha,2+2\alpha$ & $3\alpha,3$\\
    \cmidrule{2-3}
    $D$ & $3,3\alpha$ & $1+\alpha,1+\alpha$\\
    \cmidrule{2-3}
  \end{tabular}
\end{center}
Damit dieses Spiel ein Gefangenendilemma ist, muss bspw.
$T = 3 > 2+2\alpha = R \iff \alpha < \frac{1}{2}$
gelten.
Die restlichen Ungleichungen werden dadurch ebenfalls erfüllt.

Für $\alpha \geq \frac{1}{2}$ ist das Spiel kein Gefangenendilemma.
Die Nash-Gleichgewichte unterscheiden sich je nach Wert von $\alpha$.
\begin{description}
  \item[Fall $\alpha = \frac{1}{2}$:] Das Spiel besitzt folgende Auszahlungen:
    \begin{center}
      \begin{tabular}{ccc}
        & $C$ & $D$\\
        \cmidrule{2-3}
        $C$ & $3, 3$ & $\frac{3}{2},3$\\
        \cmidrule{2-3}
        $D$ & $3,\frac{3}{2}$ & $\frac{3}{2},\frac{3}{2}$\\
        \cmidrule{2-3}
      \end{tabular}
    \end{center}
    In diesem Spiel sind alle Strategiekombinationen Nash-Gleichgewichte:
    \begin{align*}
      \text{NGG} & = \{(C,C), (C,D), (D,C), (D,D)\}
    \end{align*}

  \item[Fall $\alpha > \frac{1}{2}$:] In diesem Fall lässt sich $\alpha$ darstellen als
    $\alpha = \frac{1}{2} + \beta$ mit $\beta > 0$.
    Das resultierende Spiel besitzt folgende Auszahlungen:
    \begin{center}
      \begin{tabular}{ccc}
        & $C$ & $D$\\
        \cmidrule{2-3}
        $C$ & $\underline{3+2\beta}, \underline{3+2\beta}$
            & $\underline{\frac{3}{2}+3\beta}, 3$\\
        \cmidrule{2-3}
        $D$ & $3, \underline{\frac{3}{2}+3\beta}$
            & $\frac{3}{2}+\beta, \frac{3}{2}+\beta$\\
        \cmidrule{2-3}
      \end{tabular}
    \end{center}
    Die jeweils besten Antworten sind unterstrichen, sodass $\text{NGG} = \{(C,C)\}$ gilt.
    Weiterhin wird $D$ von $C$ in diesem Spiel strikt dominiert.
\end{description}

\paragraph{Aufgabe 5}%
\label{par:serie_2_aufgabe_5}

Zwei Tiere kämpfen um eine Beute.
Jedes der Tiere kann entweder passiv oder aggressiv sein.
Jedes Tier zieht es vor aggressiv zu sein, wenn der Gegner passiv ist, und passiv zu sein,
wenn der Gegner aggressiv ist.
Gegeben das eigene Verhalten zieht es jedes Tier vor, wenn der Gegner passiv ist.
Finden Sie eine geeignete Auszahlungsfunktion, welche diese Präferenzen abbildet und
bestimmen Sie die Nash-Gleichgewichte des resultierenden Spiels.

\subparagraph{Lösung}%

Da beide Tiere das gleiche Verhalten zeigen, sind die Auszahlungsmatrizen symmetrisch und
wir können die Auszahlungen wie folgt bezeichnen:
\begin{center}
  \begin{tabular}{ccc}
    & $a$ & $p$\\
    \cmidrule{2-3}
    $a$ & $x,x$ & $z,y$\\
    \cmidrule{2-3}
    $p$ & $y,z$ & $w,w$\\
    \cmidrule{2-3}
  \end{tabular}
\end{center}
Für die Auszahlungen gelten folgende Bedingungen:
\begin{align*}
  z & > w & \text{(Wenn der Gegner passiv ist, dann ist aggressiv sein besser.)}\\
  y & > x & \text{(Wenn der Gegner aggressiv ist, dann ist passiv sein besser.)}\\
  z & > x & \text{(Gegeben das eigene Verhalten,}\\
  w & > y & \text{wird passives Verhalten vom Gegner bevorzugt.)}
\end{align*}
Dies führt zu $z > w > y > x$, sodass z.\,B. $z = 3, w = 2, y = 1$ und $x=0$ die
Bedingungen erfüllen.

Im folgenden Spiel sind die besten Antworten unterstrichen:
\begin{center}
  \begin{tabular}{ccc}
    & $a$ & $p$\\
    \cmidrule{2-3}
    $a$ & $0,0$ & $\underline{3},\underline{1}$\\
    \cmidrule{2-3}
    $p$ & $\underline{1},\underline{3}$ & $2,2$\\
    \cmidrule{2-3}
  \end{tabular}
\end{center}
Die Nash-Gleichgewichte sind somit $\text{NGG} = \{(a,p), (p,a)\}$.
