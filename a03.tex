\subsection{Serie 3}%
\label{sub:serie_3}

\paragraph{Aufgabe 1}%
\label{par:serie_3_aufgabe_1}

\begin{enumerate}
  \item Bestimmen Sie alle Nash-GG (in reinen Strategien) des folgenden Spiels in
    strategischer Form:
    \begin{center}
      \begin{tabular}{ccc}
        & $B$ & $S$\\
        \cmidrule{2-3}
        $B$ & $2,1$ & $0,0$\\
        \cmidrule{2-3}
        $S$ & $2,0$ & $1,2$\\
        \cmidrule{2-3}
      \end{tabular}
    \end{center}
  \item Bestimmen Sie alle Nash-GG (in reinen Strategien) des folgenden Spiels in
    strategischer Form in Abhängigkeit von $x$:
    \begin{center}
      \begin{tabular}{ccc}
        & $A$ & $B$\\
        \cmidrule{2-3}
        $U$ & $x,2$ & $2,0$\\
        \cmidrule{2-3}
        $D$ & $2,1$ & $4,2$\\
        \cmidrule{2-3}
      \end{tabular}
    \end{center}
  \item Bestimmen Sie alle Nash-GG (in reinen Strategien) des folgenden Spiels in
    strategischer Form:
    \begin{center}
      \begin{tabular}{ccc}
        & $L$ & $R$\\
        \cmidrule{2-3}
        $L$ & $-1, \phantom{-}1$ & $\phantom{-}1, -1$\\
        \cmidrule{2-3}
        $R$ & $\phantom{-}1, -1$ & $-1, \phantom{-}1$\\
        \cmidrule{2-3}
      \end{tabular}
    \end{center}
\end{enumerate}

\subparagraph{Lösung}%

\begin{enumerate}
  \item Die jeweils besten Antworten lauten:
    \begin{center}
      \begin{tabular}{ccc}
        & $B$ & $S$\\
        \cmidrule{2-3}
        $B$ & $\underline{2},\underline{1}$ & $0,0$\\
        \cmidrule{2-3}
        $S$ & $\underline{2},0$ & $\underline{1},\underline{2}$\\
        \cmidrule{2-3}
      \end{tabular}
    \end{center}
    Somit gilt $\text{NGG} = \{(B,B), (S,S)\}$.

  \item Unabhängig von $x$ sind folgende unterstrichenen Auszahlungen jeweils beste
    Antworten der Spieler:
    \begin{center}
      \begin{tabular}{ccc}
        & $A$ & $B$\\
        \cmidrule{2-3}
        $U$ & $x,\underline{2}$ & $2,0$\\
        \cmidrule{2-3}
        $D$ & $2,1$ & $\underline{4},\underline{2}$\\
        \cmidrule{2-3}
      \end{tabular}
    \end{center}
    Damit ist $(D,B)$ immer ein Element des NGG.

    Für die vollständige Bestimmung der Nash-Gleichgewichte bedarf es einer
    Fallunterscheidung von $x$ im Vergleich zur alternativen Auszahlung von $2$ für
    $(D,A)$.

    \begin{description}
      \item[$x<2$:] Die besten Antworten lauten:
        \begin{center}
          \begin{tabular}{ccc}
            & $A$ & $B$\\
            \cmidrule{2-3}
            $U$ & $x,\underline{2}$ & $2,0$\\
            \cmidrule{2-3}
            $D$ & $\underline{2},1$ & $\underline{4},\underline{2}$\\
            \cmidrule{2-3}
          \end{tabular}
        \end{center}
        Damit gilt: $\text{NGG} = \{(D,B)\}$.

      \item[$x=2$:] Die besten Antworten lauten:
        \begin{center}
          \begin{tabular}{ccc}
            & $A$ & $B$\\
            \cmidrule{2-3}
            $U$ & $\underline{x},\underline{2}$ & $2,0$\\
            \cmidrule{2-3}
            $D$ & $\underline{2},1$ & $\underline{4},\underline{2}$\\
            \cmidrule{2-3}
          \end{tabular}
        \end{center}
        Damit gilt: $\text{NGG} = \{(U, A), (D,B)\}$.

      \item[$x>2$:] Die besten Antworten lauten:
        \begin{center}
          \begin{tabular}{ccc}
            & $A$ & $B$\\
            \cmidrule{2-3}
            $U$ & $\underline{x},\underline{2}$ & $2,0$\\
            \cmidrule{2-3}
            $D$ & $2,1$ & $\underline{4},\underline{2}$\\
            \cmidrule{2-3}
          \end{tabular}
        \end{center}
        Damit gilt wie im Fall $x=2$: $\text{NGG} = \{(U, A), (D,B)\}$.
    \end{description}

  \item Die jeweils besten Antworten lauten:
    \begin{center}
      \begin{tabular}{ccc}
        & $L$ & $R$\\
        \cmidrule{2-3}
        $L$ & $-1, \phantom{-}\underline{1}$ & $\phantom{-}\underline{1}, -1$\\
        \cmidrule{2-3}
        $R$ & $\phantom{-}\underline{1}, -1$ & $-1, \phantom{-}\underline{1}$\\
        \cmidrule{2-3}
      \end{tabular}
    \end{center}

    Damit ist keine Strategiekombination Teil des Nash-Gleichgewichts und es gilt
    $\text{NGG} = \emptyset$.
\end{enumerate}

\paragraph{Aufgabe 2}%
\label{par:serie_3_aufgabe_2}

Bestimmen Sie für folgendes Spiel in Normalform alle Nash-Gleichgewichte in reinen
Strategien:

\begin{center}
  \begin{tabular}{ccccc}
    & & \multicolumn{3}{c}{Spieler 2}\\
    & & $L$ & $C$ & $R$\\
    \cmidrule{3-5}
    \multirow{3}{*}{Spieler 1}
    & $T$ & $0,1$ & $9,0$ & $2,3$\\
    \cmidrule{3-5}
    & $M$ & $5,9$ & $7,3$ & $1,7$\\
    \cmidrule{3-5}
    & $B$ & $7,5$ & $10,10$ & $3,5$\\
    \cmidrule{3-5}
  \end{tabular}
\end{center}

\subparagraph{Lösung}%

Für Spieler 1 dominiert $B$ die anderen beiden Strategien, sodass $T$ und $M$ eliminiert
werden können und das folgende reduzierte Spiel entsteht:
\begin{center}
  \begin{tabular}{ccccc}
    & & \multicolumn{3}{c}{Spieler 2}\\
    & & $L$ & $C$ & $R$\\
    \cmidrule{3-5}
    Spieler 1 & $B$ & $7,5$ & $10,10$ & $3,5$\\
    \cmidrule{3-5}
  \end{tabular}
\end{center}

Die beste Antwort auf $B$ von Spieler 2 ist $C$, sodass $\text{NGG} = \{(B,C)\}$ das
einzige Nash-Gleichgewicht ist.

\paragraph{Aufgabe 3}%
\label{par:serie_3_aufgabe_3}

Geben Sie jeweils ein Beispiel für die Auszahlungsmatrix eines Zwei-Personen-Spiels in
strategischer Form an, in dem jeder Spieler nur zwei reine Strategien besitzt, um zu
belegen, dass folgende Aussagen für solche Spiele \emph{falsch} sind.

\begin{enumerate}
  \item Gibt es nur ein Nash-GG in reinen Strategien, so verwenden in diesem Spiel beide
    Spieler eine dominante Strategie.
  \item Ist ein Spiel symmetrisch, so besitzt es ein symmetrisches Nash-GG in reinen
    Strategien.
\end{enumerate}

\subparagraph{Lösung}%

\begin{enumerate}
  \item Betrachte folgendes Spiel:
    \begin{center}
      \begin{tabular}{ccc}
        & $A$ & $B$\\
        \cmidrule{2-3}
        $C$ & $3,3$ & $2,2$\\
        \cmidrule{2-3}
        $D$ & $1,2$ & $3,1$\\
        \cmidrule{2-3}
      \end{tabular}
    \end{center}
    Es gilt $\text{NGG} = \{(C,A)\}$, aber nur $A$ ist eine dominante Strategie.

  \item Die Auszahlungsbimatrix eines symmetrischen Spiels besitzt folgende Form:
    \begin{center}
      \begin{tabular}{ccc}
        & $A$ & $B$\\
        \cmidrule{2-3}
        $A$ & $a,a$ & $b,c$\\
        \cmidrule{2-3}
        $B$ & $c,b$ & $d,d$\\
        \cmidrule{2-3}
      \end{tabular}
    \end{center}
    Für $a=d=0$ und $b=c=1$ gilt $\text{NGG} = \{(A,B), (B,A)\}$, welches nicht
    symmetrisch ist.
\end{enumerate}

\paragraph{Aufgabe 4}%
\label{par:serie_3_aufgabe_4}

Bei einer Wahl gibt es zwei Kandidaten, $A$ und $B$.
Es gibt $n \geq 2$ Wahlberechtigte, von denen $k \geq 1$ Kandidaten $A$ vorziehen
und $m = n-k \geq 1$ Kandidaten $B$ vorziehen.
Jeder Wahlberechtigte muss entscheiden, ob er an der Wahl teilnimmt (in diesem Falle sei
unterstellt, dass er für seinen favorisierten Kandidaten stimmt) oder nicht.
Ein Wahlberechtigter, der nicht an der Wahl teilnimmt,
erhält eine Auszahlung von $2$, falls sein favorisierter Kandidat gewinnt;
eine Auszahlung von $1$, falls beide Kandidaten gleich viele Stimmen erhalten,
und eine Auszahlung von $0$, falls sein favorisierter Kandidat verliert.
Nimmt der Wahlberechtigte an der Wahl teil, sind die Auszahlungen entsprechend, jedoch
werden Teilnahmekosten in der Höhe von $c \in (0,1)$ von allen Auszahlungen abgezogen.
\begin{enumerate}
  \item Sei $k=m=1$. Bestimmen Sie alle Nash-GG in reinen Strategien.
  \item Sie $k=m>1$. Bestimmen Sie alle Nash-GG in reinen Strategien.
  \item Sei $k<m$. Welche Nash-GG in reinen Strategien gibt es?
\end{enumerate}

\subparagraph{Lösung}%

\begin{enumerate}
  \item Jeder Spieler hat die folgenden Aktionen:
    $w$ (an der Wahl teilnehmen) und $\neg w$ (nicht an der Wahl teilnehmen).

    Die Auszahlungen mit den jeweils besten Antworten lauten wie folgt:
    \begin{center}
      \begin{tabular}{ccc}
        & $w$ & $\neg w$\\
        \cmidrule{2-3}
        $w$ & $\underline{1-c},\underline{1-c}$ & $\underline{2-c},0$\\
        \cmidrule{2-3}
        $\neg w$ & $0,\underline{2-c}$ & $1,1$\\
        \cmidrule{2-3}
      \end{tabular}
    \end{center}
    Damit gilt $\text{NGG} = \{(w,w)\}$.

  \item Wir betrachten verschiedene Situationen und entscheiden, ob sie ein
    Nash-Gleichgewicht sind oder nicht.
    Sei $n_A$ die Anzahl der Wahlberechtigten, die für $A$ an der Wahl teilnehmen und
    $n_B$ entsprechend für $B$.

    \begin{description}
      \item[$n_A = n_B = k$:] In diesem Fall wählen alle Wahlberechtigten.
        Falls ein Wahlberechtigter sich entscheiden würde doch nicht an der Wahl
        teilzunehmen, dann würde der Kandidat die Wahl verlieren und die eigene Auszahlung
        würde sich verringern.
        Daher hat kein Wahlberechtigter einen Anreiz vom Verhalten abzuweichen und die
        Situation ist ein Nash-Gleichgewicht.

      \item[$n_A = n_B < k$:] In diesem Fall ist die Wahl ein Gleichstand, aber nicht alle
        Wahlberechtigten nehmen an der Wahl teil.
        Damit hätte jedoch irgendein nichtwählender Wahlberechtigter einen Anreiz doch an
        der Wahl teilzunehmen und die eigene Auszahlung von $1$ auf $2-c$ zu verbessern.
        Daher ist diese Situation kein Nash-Gleichgewicht.

      \item[$n_A = n_B + 1$:] In diesem Fall gewinnt ein Kandidat die Wahl mit einer
        Stimme Vorsprung (analog $n_B = n_A + 1$).
        Dadurch hat ein Wahlberechtiger den Anreiz die Wahl zum Gleichstand zu führen und
        die eigene Auszahlung von $0$ auf $1-c$ zu verbessern.
        Daher ist diese Situation kein Nash-Gleichgewicht.

      \item[$n_A \geq n_B + 2$:] In diesem Fall gewinnt ein Kandidat die Wahl mit mehr als
        einer Stimme Vorsprung (analog $n_B \geq n_A+2$).
        Dadurch hat ein für den Gewinner wählender Wahlberechtigter den Anreiz nicht an
        der Wahl teilzunehmen und die eigene Auszahlung von $1-c$ auf $2-c$ zu verbessern.
        Daher ist diese Situation kein Nash-Gleichgewicht.
    \end{description}
    Damit ist das einzige Nash-Gleichgewicht \emph{alle wählen}.

  \item Analog zur vorherigen Teilaufgabe betrachten wir verschiedene Situationen und
    unterscheiden, ob sie Nash-Gleichgewichte sind.
    Sei wieder $n_A$ die Anzahl der Wahlberechtigten, die für $A$ an der Wahl teilnehmen
    und $n_B$ entsprechend für $B$.

    \begin{description}
      \item[$n_A = n_B \leq k$:] In diesem Fall haben alle Wahlberechtigten für $A$ an der
        Wahl teilgenommen und es herrscht Gleichstand.
        Dann hätte ein nichtwählender Wahlberechtigter für $B$ den Anreiz doch an der Wahl
        teilzunehmen und die Auszahlung von $1$ auf $2-c$ zu verbessern.
        Daher ist diese Situation kein Nash-Gleichgewicht.

      \item[$n_A = n_B + 1$:] In diesem Fall gewinnt ein Kandidat die Wahl mit einer
        Stimme Vorsprung (analog $n_B = n_A + 1$).
        Dadurch hat ein Wahlberechtiger den Anreiz die Wahl zum Gleichstand zu führen und
        die eigene Auszahlung von $0$ auf $1-c$ zu verbessern.
        Daher ist diese Situation kein Nash-Gleichgewicht.

      \item[$n_A \geq n_B + 2$:] In diesem Fall gewinnt ein Kandidat die Wahl mit mehr als
        einer Stimme Vorsprung (analog $n_B \geq n_A+2$).
        Dadurch hat ein für den Gewinner wählender Wahlberechtigter den Anreiz nicht an
        der Wahl teilzunehmen und die eigene Auszahlung von $1-c$ auf $2-c$ zu verbessern.
        Daher ist diese Situation kein Nash-Gleichgewicht.
    \end{description}

    Damit existieren keine Nash-Gleichgewichte für den Fall $k<m$.
\end{enumerate}
