\subsection{Serie 4}%
\label{sub:serie_4}

\paragraph{Aufgabe 1: Bertrand Duopol}%
\label{par:aufgabe_4_1}

Betrachten Sie 2 Unternehmen, die ein homogenes Gut zu identischen Stückkosten $c$
produzieren können.
Beide Unternehmen stehen im Preiswettbewerb miteinander, d.\,h. Unternehmen $i \in \{1,
2\}$ wählen gleichzeitig Preis $p_i \in \RealNumbers_+$, um ihren Gewinn zu maximieren.
Alle Konsumenten kaufen beim Anbieter mit dem niedrigsten Preis, der die gesamte Nachfrage
$q(p_i, p_j)$ zu diesem Preis bedienen muss.
Bei gleichen Preisen teilen sich die Konsumenten $50:50$ auf.
Die individuelle Nachfrage des Unternehmens $i \in \{1, 2\}$ ist also gegeben durch:
\begin{align*}
  q_i(p_i, p_j) =
  \begin{cases}
    q(p_i, p_j) & \text{wenn } p_i < p_j\\
    \frac{1}{2} q(p_i, p_j) & \text{wenn } p_i = p_j\\
    0 & \text{wenn } p_i > p_j
  \end{cases}
\end{align*}
Die Gewinnfunktion von Unternehmen $i \in \{1, 2\}$ ist dann:
\begin{align*}
  \pi_i(p_i, p_j) =
  \begin{cases}
    (p_i - c)q(p_i, p_j) & \text{wenn } p_i < p_j\\
    (p_i - c)\frac{1}{2} q(p_i, p_j) & \text{wenn } p_i = p_j\\
    0 & \text{wenn } p_i > p_j
  \end{cases}
\end{align*}
Bestimmen Sie das eindeutige Nash-GG in reinen Strategien.

\paragraph{Aufgabe 2: Cournot Duopol}%
\label{par:aufgabe_2_cournot_duopol}

Betrachten Sie erneut eine Situation mit 2 Unternehmen, die ein homogenes Gut zu
identischen und konstanten Grenzkosten produzieren $c_1 = c_2 = c$.
Es entstehen keine Fixkosten.
Die Unternehmen wählen nun simultan ihre individuellen Produktionsmengen $x_1, x_2 \in
\RealNumbers_+$.
Der Marktpreis ergibt sich dann aus dem aggregierten Angebot $x = x_1 + x_2$ und der
inversen Nachfragefunktion:
\begin{align*}
  P(x) & = \begin{cases}
    a - b \cdot x & \text{wenn } 0 \leq x < \frac{a}{b}\\
    0 & \text{wenn } x \geq \frac{a}{b}\\
  \end{cases}
\end{align*}
Bestimmen Sie die Gewinnfunktion von Unternehmen $i \in \{1,2\}$ und das eindeutige
Nash-Gleichgewicht in reinen Strategien.
