\subsection{Serie 5}%
\label{sub:serie_5}

\paragraph{Aufgabe 1}%
\label{par:serie_5_aufgabe_1}

Betrachten Sie folgendes Spiel in strategischer Form:
\begin{center}
  \begin{tabular}{ccccc}
    & & \multicolumn{3}{c}{Spieler 2}\\
    & & $L$ & $C$ & $R$\\
    \cmidrule{3-5}
    \multirow{3}{*}{Spieler 1}
    & $T$ & $4,2$ & $4,3$ & $1,1$\\
    \cmidrule{3-5}
    & $M$ & $5,3$ & $6,4$ & $2,4$\\
    \cmidrule{3-5}
    & $B$ & $6,4$ & $4,2$ & $2,5$\\
    \cmidrule{3-5}
  \end{tabular}
\end{center}

Identifizieren Sie die Strategien, die IESDS überstehen, und bestimmen Sie alle
Nash-Gleichgewichte des dargestellten Spiels.

\subparagraph{Lösung}%

$T$ wird strikt von $M$ dominiert und daher eliminiert.
\begin{center}
  \begin{tabular}{ccccc}
    & & \multicolumn{3}{c}{Spieler 2}\\
    & & $L$ & $C$ & $R$\\
    \cmidrule{3-5}
    \multirow{2}{*}{Spieler 1}
    & $M$ & $5,3$ & $6,4$ & $2,4$\\
    \cmidrule{3-5}
    & $B$ & $6,4$ & $4,2$ & $2,5$\\
    \cmidrule{3-5}
  \end{tabular}
\end{center}
$L$ wird strikt von $R$ dominiert und eliminiert.
\begin{center}
  \begin{tabular}{cccc}
    & & \multicolumn{2}{c}{Spieler 2}\\
    & & $C$ & $R$\\
    \cmidrule{3-4}
    \multirow{2}{*}{Spieler 1}
    & $M$ & $\underline{6},\underline{4}$ & $\underline{2},\underline{4}$\\
    \cmidrule{3-4}
    & $B$ & $4,2$ & $\underline{2},\underline{5}$\\
    \cmidrule{3-4}
  \end{tabular}
\end{center}
In reinen Strategien erhalten wir folgendes Nash-Gleichgewicht:
$\text{NGG} = \{(M,C), (M,R), (B,R)\}$.

Betrachten wir nun gemischte Strategien.
Spieler 1 spielt $M$ mit der Wahrscheinlichkeit $p$
und $B$ mit der Wahrscheinlichkeit $1-p$.
Spieler 2 spielt $C$ mit der Wahrscheinlichkeit $q$
und $R$ mit der Wahrscheinlichkeit $1-q$.
Der Erwartungsnutzen der Spieler hängt von Wahrscheinlichkeit des Gegenspielers ab eine
bestimmte Strategie zu spielen.
Für Spieler 1 sind die einzelnen Erwartungsnutzen:
\begin{align*}
  EU_1(M) & = 6q + 2(1-q)\\
  EU_1(B) & = 4q + 2(1-q)
\end{align*}
Spieler 2 muss $q$ so wählen, dass Spieler 1 indifferent zwischen $M$ und $B$ ist und
damit keinen Vorteil aus dem Verhalten von Spieler 2 ziehen kann,
d.\,h. es muss $EU_1(M) = EU_1(B)$ gelten.
\begin{align*}
  EU_1(M) = EU_1(B) \iff 6q + 2(1-q) = 4q + 2(1-q)  \iff q = 0
\end{align*}
Analog gilt für Spieler 2:
\begin{align*}
  EU_2(C) & = 4p + 2(1-p)\\
  EU_2(R) & = 4p + 5(1-p)\\
  EU_2(C) & = EU_2(R) \iff 4p + 2(1-p) = 4p + 5(1-p) \iff p = 1
\end{align*}
Damit ist $\text{NGG} = \{(M,R)\}$ das einzige Nash-Gleichgewicht in gemischten
Strategien.
Dies wird auch dadurch begründet, dass $B$ von $M$ schwach dominiert wird und $C$ von $R$
ebenso schwach dominiert wird.

\paragraph{Aufgabe 2}%
\label{par:serie_5_aufgabe_2}

Betrachten Sie das folgende \emph{Gefangenendilemma} und stellen Sie die
Reaktionsfunktionen der Spieler graphisch dar.
\begin{center}
  \begin{tabular}{ccc}
    & $c$ & $d$\\
    \cmidrule{2-3}
    $c$ & $3,\phantom{-}3$ & $-1,4$\\
    \cmidrule{2-3}
    $d$ & $4,-1$ & $\phantom{-}0,0$\\
    \cmidrule{2-3}
  \end{tabular}
\end{center}

\subparagraph{Lösung}%

Sei $p$ die Wahrscheinlichkeit, dass Spieler 1 $d$ spielt,
und $q$ die Wahrscheinlichkeit, dass Spieler 2 $d$ spielt.

\begin{center}
  \begin{tikzpicture}
    \draw[->] (0,0) -- (6.5,0);
    \draw (6,-0.1) -- (6,0.1);
    \draw[->] (0,0) -- (0,6.5);
    \draw (-0.1,6) -- (0.1,6);

    \draw (0,0) [below] node {$0$};
    \draw (6,0) [below] node {$1$};
    \draw (6.5,0) [right] node {$p$};
    \draw (0,0) [left] node {$0$};
    \draw (0,6) [left] node {$1$};
    \draw (0,6.5) [above] node {$q$};

    \draw [very thick, red] (0,6) -- (6,6);
    \draw [very thick, blue] (6,0) -- (6,6);
    \fill (6,6) circle (2pt);
  \end{tikzpicture}
\end{center}
Im Gefangenendilemma wird $c$ strikt dominiert, daher spielen beide Spieler immer $d$.

\paragraph{Aufgabe 3}%
\label{par:serie_5_aufgabe_3}

Das im Folgenden dargestellte Spiel nennt sich \emph{Chicken}; es handelt sich um eine
Variante des Spiels \emph{Hawk-Dove} aus der Vorlesung.
Stellen Sie auch hier die Reaktionsfunktionen der Spieler graphisch dar und bestimmen Sie
alle Nash-Gleichgewichte des Spiels.
\begin{center}
  \begin{tabular}{ccc}
    & $h$ & $d$\\
    \cmidrule{2-3}
    $h$ & $2,2$ & $\phantom{-}0,\phantom{-}4$\\
    \cmidrule{2-3}
    $d$ & $4,0$ & $-1,-1$\\
    \cmidrule{2-3}
  \end{tabular}
\end{center}

\subparagraph{Lösung}%

In reinen Strategien sind $(d,h)$ und $(h,d)$ Nash-Gleichgewichte.

Sei $p$ die Wahrscheinlichkeit, dass Spieler 1 $h$ spielt,
und $q$ die Wahrscheinlichkeit, dass Spieler 2 $h$ spielt.

Für die Erwartungsnutzen der Spieler gilt:\\
\begin{minipage}[t]{0.45\linewidth}
  \centering
  \begin{align*}
    EU_1(h) & = 2q\\
    EU_1(d) & = 4q-1(1-q)\\
    EU_1(h) & > EU_1(d) \iff q < \frac{1}{3} \implies p = 1\\
    EU_1(h) & = EU_1(d) \iff q = \frac{1}{3}\\
    EU_1(h) & < EU_1(d) \iff q > \frac{1}{3} \implies p = 0
  \end{align*}
\end{minipage}
\begin{minipage}[t]{0.45\linewidth}
  \centering
  \begin{align*}
    EU_2(h) & = 2p\\
    EU_2(d) & = 4p-1(1-p)\\
    EU_2(h) & > EU_2(d) \iff p < \frac{1}{3} \implies q = 1\\
    EU_2(h) & = EU_2(d) \iff p = \frac{1}{3}\\
    EU_2(h) & < EU_2(d) \iff p > \frac{1}{3} \implies q = 0
  \end{align*}
\end{minipage}

\begin{center}
  \begin{tikzpicture}
    \draw[->] (0,0) -- (6.5,0);
    \draw (6,-0.1) -- (6,0.1);
    \draw[->] (0,0) -- (0,6.5);
    \draw (-0.1,6) -- (0.1,6);

    \draw (0,0) [below] node {$0$};
    \draw (6,0) [below] node {$1$};
    \draw (6.5,0) [right] node {$p$};
    \draw (0,0) [left] node {$0$};
    \draw (0,6) [left] node {$1$};
    \draw (0,6.5) [above] node {$q$};

    \draw (-0.1,2) -- (0.1,2);
    \draw (0,2) -- (0,2) [left] node {$\frac{1}{3}$};
    \draw (2,-0.1) -- (2,0.1);
    \draw (2,0) -- (2,0) [below] node {$\frac{1}{3}$};

    \draw [very thick, blue]  (6,0) -- (6,2);
    \draw [very thin, blue]   (6,2) -- (0,2);
    \draw [very thick, blue]  (0,2) -- (0,6);

    \draw [very thick, red]   (0,6) -- (2,6);
    \draw [very thin, red]   (2,6) -- (2,0);
    \draw [very thick, red]    (2,0) -- (6,0);

    \fill (2,2) circle (2pt);
    \fill (6,0) circle (2pt);
    \fill (0,6) circle (2pt);
  \end{tikzpicture}
\end{center}
In gemischten Strategien ist daher
$\text{NGG} = \left\{(1,0), (\frac{1}{3}, \frac{1}{3}), (0,1)\right\}$.

\paragraph{Aufgabe 4}%
\label{par:serie_5_aufgabe_4}

Betrachten Sie die folgenden Varianten des Cournot-Duopols aus der Vorlesung.
In beiden Varianten ist die inverse Marktnachfragefunktion $P(q) = \max\{1-q, 0\}$.
\begin{enumerate}
  \item Die Firmen haben konstante aber unterschiedliche Stückkosten:
    die Kostenfunktion von Firma 1 ist $C_1(q_1) = c_1 q_1$,
    die Kostenfunktion von Firma 2 ist $C_2(q_2) = c_2 q_2$
    mit $1 > c_1 > c_2 > 0$.
    Bestimmen Sie die besten Antworten beider Firmen und die Nash-Gleichgewichte.
  \item Die Firmen haben identische Kostenfunktionen,
    die für $q_i > 0$ durch $C_i(q_i) = f+c q_i$ mit $f > 0$ und $c < 1$ gegeben sind.
    Für $q_i = 0$ gilt $C_i(q_i) = 0$ (d.\,h. $f$ stellt quasi-fixe Kosten dar, die genau
    dann anfallen, wenn die Firmat die Produktion aufnimmt).
    Welche Bedingung an die Parameter $(c,f)$ muss erfüllt sein,
    damit $(q_1^*, q_2^*) = (0,0)$ ein Nash-Gleichgewicht ist?
    Für welche Werte der Parameter gibt es ein Nash-Gleichgewicht, in dem nur eine Firma
    eine strikt positive Menge $(q_i^* > 0, q_{-i}^* = 0)$ wählt?
    Welche Bedingung muss erfüllt sein, damit es ein Nash-Gleichgewicht gibt, in dem beide
    Firmen strikt positive Mengen wählen?
\end{enumerate}
