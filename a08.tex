\subsection{Serie 8}%
\label{sub:serie_8}

\paragraph{Aufgabe 1}%
\label{par:serie_8_aufgabe_1}

Spieler 1 und 2 wechseln sich in einem Spiel mit vollkommener Information darin ab
(beginnend mit Spieler 1), Steine von einem Haufen wegzunehmen.
Zu Beginn des Spiels sind $n > 0$ Steine auf dem Haufen.
Ein Spieler, der am Zug ist, kann einen oder zwei Steine wegnehmen.
Das Spiel endet, wenn kein Stein mehr auf dem Haufen liegt.
Der Spieler, der den letzten Stein wegnimmt, ist der Gewinner.
Die Auszahlung des Gewinners ist 1; die des Verlierers ist 0.
Für $n = 1$ und $n = 2$ gewinnt (in einem teilspielperfekten Gleichgewicht) offenkundig
Spieler 1 dieses Spiel.

\begin{enumerate}
  \item Wer gewinnt für $n = 3$? Wer für $n = 4$?
  \item Können Sie den Gewinner für beliebiges $n$ bestimmen?
\end{enumerate}

\subparagraph{Lösung}%

\begin{enumerate}
  \item Für $n=3$:
    \begin{center}
      \begin{tikzpicture}[
          level/.style={sibling distance=60mm/#1},
        ]
        \node {$1_{\color{mygray}{3}}$}
          child {
            node {$2_{\color{mygray}{2}}$}
            child {
              node {$1_{\color{mygray}{1}}$}
              child {
                node {$1,0$}
              }
              edge from parent node [above left] {1}
            }
            child [very thick] {
              node {$0, 1$}
              edge from parent node [above right] {2}
            }
            edge from parent node [above left] {1}
          }
          child {
            node {$2_{\color{mygray}{1}}$}
            child [very thick] {
              node {$0,1$}
              edge from parent node [right] {1}
            }
            edge from parent node [above right] {2}
          }
        ;
      \end{tikzpicture}
    \end{center}
    Bei $n=3$ gewinnt immer Spieler 2.

    Für $n=4$:
    \begin{center}
      \begin{tikzpicture}[
          level/.style={sibling distance=60mm/#1},
        ]
        \node {$1_{\color{mygray}{4}}$}
          child {
            node {$2_{\color{mygray}{3}}$}
            child {
              node {$1_{\color{mygray}{2}}$}
              child [very thick] {
                node {$1,0$}
              }
              edge from parent node [above left] {1}
            }
            child {
              node {$1_{\color{mygray}{1}}$}
              child {
                node {$1,0$}
                edge from parent node [right] {1}
              }
              edge from parent node [above right] {2}
            }
            edge from parent node [above left] {1}
          }
          child {
            node {$2_{\color{mygray}{2}}$}
            child [very thick] {
              node {$0,1$}
            }
            edge from parent node [above right] {2}
          }
        ;
      \end{tikzpicture}
    \end{center}
    Bei $n=4$ gewinnt immer Spieler 1 mit der Strategie 11 bzw. 12.

\end{enumerate}


\paragraph{Aufgabe 2}%
\label{par:serie_8_aufgabe_2}

In einem Spiel mit vollkommener Information gibt es drei Felder: $A, B$ und $C$.
Spieler 1 beginnt das Spiel, indem er auf die Felder positive Geldbeträge
$(x_A, x_B, x_C)$ legt.
Danach ist Spieler 2 am Zug und kann positive Geldbeträge $(y_A, y_B, y_C)$ auf die Felder
legen.
Damit endet das Spiel.
Der Gewinner ist der Spieler, der die meisten Felder gewinnt.
Spieler 1 gewinnt das Feld $f \in \{A,B,C\}$, wenn $x_f > y_f$ gilt;
ansonsten gewinnt Spieler 2 das Feld.
Gewinnt Spieler $i$, erhält er einen Preis in Höhe von $V_i > 0$.
Jeder Spieler muss die Summe seiner Gebote bezahlen.
Wenn Spieler 1 gewinnt, ist seine Auszahlung also
\begin{align*}
  V_1 - x_A - x_B - x_C,
\end{align*}
wenn er verliert, ist sie
\begin{align*}
  - x_A - x_B - x_C,
\end{align*}
Die Auszahlungen von Spieler 2 sind entsprechend.

\begin{enumerate}
  \item Bestimmen Sie das (jeweils eindeutig bestimmte) Ergebnis der teilspielperfekten
    Gleichgewichte für den Fall $V_1 = V_2 = 300$.
    (Hinweis: eine vollständige Bestimmung der Gleichgewichtsstrategien ist hierfür nicht
    erforderlich; Sie dürfen in der Bearbeitung unterstellen, dass es teilspielperfekte
    Gleichgewichte gibt.)
  \item Wiederholen Sie die Aufgabe für den Fall $V_1 = 300, V_2 = 100$.
\end{enumerate}

\paragraph{Aufgabe 3}%
\label{par:serie_8_aufgabe_3}

Betrachten Sie das folgende Verhandlungsspiel.
In Runde 1 macht Spieler 1 ein Angebot $x \in [0, 1]$; daraufhin entscheidet Spieler 2, ob
er das Angebot annimmt oder ablehnt.
Wenn er ablehnt, dann beginnt Runde 2 und Spieler 2 macht ein Angebot, welches Spieler 1
entweder annimmt oder ablehnt.
Einigen sich die Spieler in Runde $t$ auf ein Angebot $x_t$, erhalten sie die Payoffs
\begin{align*}
  (\alpha^t x_t, \beta^t (1-x_t)) \text{ mit $\alpha, \beta \in (0,1)$.}
\end{align*}
Das Spiel endet in Periode $T=3$, wenn das Angebot nicht angenommen wird, mit einer
Auszahlung von $(0,0)$.

\begin{enumerate}
  \item Bestimmen Sie das SPNE.
  \item Welche Auszahlungen erhalten die Spieler?
  \item Berechnen Sie die Auszahlungen für
    $(\alpha, \beta) = (0.8, 0.9)$ und
    $(\alpha, \beta) = (0.9, 0.8)$.
\end{enumerate}
